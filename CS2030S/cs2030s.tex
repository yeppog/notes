\documentclass[10pt,landscape,a4paper]{article}
\usepackage[utf8]{inputenc}
\usepackage{listings}
\usepackage{color}
\usepackage[ngerman]{babel}
\usepackage[T1]{fontenc}
%\usepackage[LY1,T1]{fontenc}
%\usepackage{frutigernext}
%\usepackage[lf,minionint]{MinionPro}
\usepackage{tikz}
\usetikzlibrary{shapes,positioning,arrows,fit,calc,graphs,graphs.standard}
\usepackage[nosf]{kpfonts}
\usepackage[t1]{sourcesanspro}
\usepackage{multicol}
\usepackage{wrapfig}
\usepackage[top=5mm,bottom=5mm,left=5mm,right=5mm]{geometry}
\usepackage[framemethod=tikz]{mdframed}
\usepackage{microtype}
\usepackage{pdfpages}
\usepackage{titlesec}


\usepackage{graphicx}

\titleformat*{\section}{\large\bfseries}
\titleformat*{\subsection}{\normalsize\bfseries}
\titleformat*{\subsubsection}{\small\bfseries}

\titlespacing*{\section}
{0pt}{1.5ex plus 0.2ex}{1.5ex plus .2ex}
\titlespacing*{\subsection}
{0pt}{1.5ex plus .2ex}{1.5ex plus .2ex}
\titlespacing*{\subsubsection}
{0pt}{1.5ex plus .2ex}{1.5ex plus .2ex}

\graphicspath{ {./images} }

\newtheorem{definition}{Definition}[section]

\definecolor{dkgreen}{rgb}{0,0.6,0}
\definecolor{gray}{rgb}{0.5,0.5,0.5}
\definecolor{mauve}{rgb}{0.58,0,0.82}

\lstset{frame=tb,
  language=Java,
  aboveskip=3mm,
  belowskip=3mm,
  showstringspaces=false,
  columns=flexible,
  basicstyle={\small\ttfamily},
  numbers=none,
  numberstyle=\tiny\color{gray},
  keywordstyle=\color{blue},
  commentstyle=\color{dkgreen},
  stringstyle=\color{mauve},
  breaklines=true,
  breakatwhitespace=true,
  tabsize=3,
  basicstyle = \footnotesize
}

\let\bar\overline

\begin{document}
\footnotesize
\begin{multicols*}{4}

\section{Liskov's Substitution Principle}
\begin{definition}
    Let $\phi(x)$ be a property provable about objects $x$ of type $T$. Then $\phi(x)$ should be true for objects $y$ of type $S$ where $S<:T$.
\end{definition}

\section{Abstract Classes}
\begin{definition}
    A class with abstract methods that cannot be instantiated directly, and has to be extended from.
\end{definition}
Abstract classes uses the keyword \texttt{abstract} for its abstract methods.

\section{Interfaces}
\begin{definition}
    An interface is a type and models what an entity can do.
\end{definition}
If class $C$ implements an interface $I$, $C<:I$ where $C$ is a sub-type of $I$.

\subsection{Impure Interfaces}
\begin{definition}
    An interface that has concrete implementations using the \texttt{default} keyword. Classes that implemenet this interface will all have this default method unless it is overidden.
\end{definition}


\section{Wrapper}
\subsection{Auto-boxing and unboxing}
\begin{definition}
    Automatic conversion between wrapper types and primitive types, such as \texttt{Integer} to \texttt{int}.
\end{definition}

\subsection{Performance}
Poorer performance as more memory is required to instantiate and collect garbage.


\section{Immutability}
\begin{definition}
    A variable or class is immutable if it has the \texttt{final} keyword. It disallows mutation or inheritance.
\end{definition}
Advantages for immutability:
\begin{itemize}
    \item \textbf{Ease of understanding.}
        \subitem Ensures that there is no mutation along the way when an object is passed around many times.
    \item \textbf{Enabling safe sharing of objects.}
        \subitem Objects can be reused without instantiating new objects.
    \item \textbf{Enables safe sharing of internals}
        \subitem \texttt{Varargs} or \texttt{T...} allows for the passing of variable arguments to as a parameter. It is as good as passing in an array. 
    \item \textbf{Enable safe concurrent execution.}
        \subitem Important when threads are running in parallel to avoid side effects and unintended mutations or an incorrect sequence of mutation to occur.
\end{itemize}
\section{Nested Classes}
\begin{definition}
    A class defined within another containing class.
\end{definition}
\begin{itemize}
    \item Nested classes \textbf{can} access private fields of the container class.
    \item Can either be \texttt{static} or \texttt{non-static}.
    \item Static nested classes are associated with the \textbf{containing class} and can only access static variables and methods.
    \item \textbf{Qualified State:} A \texttt{this} reference with a prefix of the enclosing class. (e.g \texttt{A.this.x}). 
\end{itemize}

\subsection{Local Class}
\begin{definition}
    Class defined locally within a function.
\end{definition}

\begin{lstlisting}
    void sortNames(List<String> names) {
        class NameComparator implements Comparator<String> {
            public int compare(String s1, String s2) {
                return s1.length() - s2.length();
            }
        }
        names.sort(new NameComparator());
    }
\end{lstlisting}

\subsection{Variable Capture}
\begin{definition}
    Local classes makes a copy of local variables and captures the local variable. Local classes can only access \texttt{final} variables.
\end{definition}

\subsection{Anonymous Class}
\begin{definition}
    A class that is declared and instantiated in a single statement.
\end{definition}

\begin{lstlisting}
    names.sort(new Comparator<String>() {
        public int compare(String s1, String s2) {
            return s1.length() - s2.length();
        }
    });
\end{lstlisting}

\section{Functions}
We can refer to functions in classes with the \texttt{::} operator (e.g \texttt{List::sort}).
\subsection{Pure Functions}
\begin{itemize}
    \item A pure function does not cause any side effects. It must \textbf{only} return a value given an input, and do nothing else.
    \item A pure function must be deterministic. A function must \textbf{always} produce the same output given the same input.
\end{itemize}

\subsection{Lambda Functions}
If we use the \texttt{\{\}} brackets after the \texttt{->} arrow, we need to specify which line to \texttt{return}. \\
There is no need to specify the \texttt{type} when we specify a function interface, such as \texttt{Function<S, T>} or \texttt{BiFunction<S,T,R>}.

\section{Streams}
\begin{itemize}
    \item \texttt{Predicate<T>::test}
    \item \texttt{Supplier<T>::get}
    \item \texttt{Function<T,R>::apply}
    \item \texttt{UnaryOp<T>::apply}
    \item \texttt{BiFunction<S,T,R>::apply}
\end{itemize}
\textbf{Stream Operations:} \\
\begin{tabular}{p{1.5cm}p{4.8cm}}
    \verb!forEach!  &   Applies a lambda to each element. Terminal. \\
    \verb!flatMap!  &   Applies a function and transforms into another stream but flattens it.  \\
    \verb!limit!    &   Returns a trunctated stream with the first \texttt{n} elements. \\
    \verb!takeWhile!    &   Returns a truncated stream up till the first fails the input predicate. \\
    \verb!peek!     &   Takes in a consumer and returns a new stream with that operation. \\
\end{tabular}

\section{Monad}
\textbf{Laws:}
\begin{itemize}
    \item Identity Law
    \item Associative Law
\end{itemize}

\subsection{Identity Law}
\textbf{Left Identity Law:} \texttt{Monad.of(x).flatMap(x -> f(x))} = \texttt{f(x)}. \\
\textbf{Right Identity Law:} \texttt{monad.flatMap(x -> Monad.of(x))} = \texttt{monad}.

\subsection{Associative Law}
\texttt{monad.flatMap(x -> f(x)).flatMap(x -> g(x)))} = \texttt{monad.flatMap(x -> f(x).flatMap(y -> g(y)))}.

\subsection{Functors}
\begin{definition}
     A functor is a simpler construction than a monad in that it only ensures lambdas can be applied sequentially to the value, without worrying about side information.
\end{definition}
\textbf{Laws:}
\begin{itemize}
    \item \texttt{functor.map(x -> x) == functor}.
    \item \texttt{functor.map(x -> f(x)).map(x -> g(x)) == functor.map(x -> g(f(x))}.
\end{itemize}

\section{Parallel Streams}
All parallel programs are concurrent, but not all concurrent programs are parallel. \\
\texttt{parallel()} can be called on streams to parallelise it. The opposite call is \texttt{sequential()}. The latest call overrides all the previous calls.\\

\subsection{Parallelis-ability}
\subsubsection{Interference}
\begin{definition}
    Interference means that one of the stream operations modifies the source of the stream during the execution of the terminal operation.
\end{definition}

\begin{lstlisting}
    List<String> list = new ArrayList<>(List.of("Luke", "Leia", "Han"));
    list.stream()
        .peek(name -> {
            if (name.equals("Han")) {
            list.add("Chewie"); // they belong together
            }
        })
        .forEach(i -> {});
\end{lstlisting}

\subsubsection{Stateful vs Stateless}
\begin{definition}
    A stateful lambda is one where the result depends on any state that might change during the execution of the stream.
\end{definition}

\begin{lstlisting}
    Stream.generate(scanner::nextInt)
        .map(i -> i + scanner.nextInt())
        .forEach(System.out::println)
\end{lstlisting}


\subsubsection{Side Effects}
In the code below, \texttt{forEach} modifies the arrayList and an incorrect sequence may arise.
\begin{lstlisting}
    List<Integer> list = new ArrayList<>(
        Arrays.asList(1,3,5,7,9,11,13,15,17,19));
    List<Integer> result = new ArrayList<>();
    list.parallelStream()
        .filter(x -> isPrime(x))
        .forEach(x -> result.add(x));
\end{lstlisting}
\textbf{Solution:} Use thread-safe methods or data structures like \texttt{.collect} or \texttt{CopyOnWriteArrayList<>}.
\subsubsection{Associativity}
\texttt{reduce} is parallelisable, but the function must be associative. Consider:
\begin{lstlisting}
    Stream.of(1,2,3,4).reduce(1, (x, y) -> 1 * x + y)
\end{lstlisting}
\textbf{Rules:}
\begin{itemize}
    \item \texttt{combiner.apply(identity, i) == i}
    \item \texttt{combiner} and \texttt{accumulator} must be associative.
    \item \texttt{combiner} and \texttt{accumulator} must be compatible -- \texttt{combiner.apply(u, accumulator.apply(identity, t)) == accumulator.apply(u, t)}.
\end{itemize}

\subsection{Order}
\begin{itemize}
    \item \texttt{findFirst}, \texttt{limit} and \texttt{skip} is expensive on an ordered stream as coordination is required to maintain order. \\
    \item \texttt{unordered()} can be called on streams where order is not important to speed up the parallelisation.
\end{itemize}

\section{Threads}
Java has a class \texttt{java.lang.Thread} that can be used to encapsulate a function to run in a seperate thread.
\begin{lstlisting}
    new Thread(() -> {
      for (int i = 1; i < 100; i += 1) {
        System.out.print("_");
      }
    }).start();

    new Thread(() -> {
      for (int i = 2; i < 100; i += 1) {
        System.out.print("*");
      }
    }).start();
\end{lstlisting}
The \texttt{.parallel()} call splits a stream operation into multiple threads. \\
\texttt{Thread.sleep()} can be called on a thread to add a delay to a thread before it is ready to run again.


\section{Async}
\texttt{Thread} is a low-level abstraction that is not very easy to use, a higher level abstraction would be \texttt{CompletableFuture<T>}.
\begin{tabular}{p{2cm}p{4cm}}
    \verb!completedFuture!  &   Creates a task that is completed, returns \texttt{T}. \\
    \verb!runAsync!         &   Takes in \texttt{Runnable} and returns \texttt{CompletableFuture<Void>}. \\
    \verb!supplyAsync!      &   Takes in \texttt{Supplier<T>} and returns \texttt{CompletableFuture<T>}. \\
    \verb!thenApply!        &   Map function that runs when the \texttt{CompletableFuture} instance is completed. \\
    \verb!thenCompose!      &   flatMap function that runs when the \texttt{CompletableFuture} instance is completed. \\
    \verb!thenCombine!      &   combine function that runs when the \texttt{CompletableFuture} instance is completed. \\ 
    \verb!get!              &   Blocks all operations until the given \texttt{CompletableFuture} is completed and returns the value. \\
    \verb!join!             &   Same as \texttt{get} but no checked exception is thrown. \\
    \verb!exceptionally!    &   Takes in a function that takes a \texttt{Throwable} and returns a new type \texttt{S}. Returns \texttt{CompletableFuture<S>}. \\
\end{tabular}

\section{Fork and Join}
Java has an implementation \texttt{ForkJoinPool} that is fine-tuned for the fork-join model. \\
It is a parallel divide-and-conquer mode of computation. There is a \texttt{compute} method to implement. The class \texttt{RecursiveTask<T>} supports the methods \texttt{fork}, \texttt{join} and \texttt{compute}.
\begin{lstlisting}
    class Summer extends RecursiveTask<Integer> {
        private static final int FORK_THRESHOLD = 2;
        private int low;
        private int high;
        private int[] array;

        public Summer(int low, int high, int[] array) {
            this.low = low;
            this.high = high;
            this.array = array;
        }

        @Override
        protected Integer compute() {
            // stop splitting into subtask if array is already small.
            if (high - low < FORK_THRESHOLD) {
                int sum = 0;
                for (int i = low; i < high; i++) {
                sum += array[i];
                }
                return sum;
            }   

            int middle = (low + high) / 2;
            Summer left = new Summer(low, middle, array);
            Summer right = new Summer(middle, high, array);
            left.fork();
            return right.compute() + left.join();
        }
    }
\end{lstlisting}
\begin{itemize}
    \item Each thread has a queue of tasks.
    \item If a thread is idle, it checks the queue and picks a task at the had and executes it. If the queue is empty, it finds another queue that is non-empty to dequeue from.
    \item When \texttt{fork} is called, the caller adds itself to the head of the queue 
    \item When \texttt{join} is called:
        \subitem If the subtask to join has not been executed, then \texttt{compute} is called on the subtask.
        \subitem If the subtask is completed, then the result is read and \texttt{join} returns.
\end{itemize}

\subsection{Order of fork() and join()}
We should \texttt{join()} the most recently \texttt{fork()}-ed task first since \texttt{ForkJoinPool} adds and removes tasks from the queue in the order in which we call \texttt{fork} and \texttt{join}. \\
\textbf{Fast Example:}
\begin{lstlisting}
    left.fork();
    right.fork();
    return right.join() + left.join();
\end{lstlisting}
\textbf{Slow Example:}
\begin{lstlisting}
    left.fork();
    right.fork();
    return left.join() + right.join();
\end{lstlisting}




\end{multicols*}
\end{document}